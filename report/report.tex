\documentclass{article}

\usepackage[english]{babel}

\usepackage{biblatex}
\usepackage{hyperref}
\usepackage{url}
\usepackage{semantic}
\usepackage{amsmath}
\usepackage{amssymb}
\usepackage{stmaryrd}
\usepackage{csquotes}

\newtheorem{theorem}{Theorem}

\bibliography{report}

\newcommand{\lambdach}{$\lambda_{ch}$}
\newcommand{\sco}[1]{\textsf{#1}}
\newcommand{\lam}[3]{\backslash #1 : #2, #3}
\newcommand{\unit}{\sco{unit}}
\newcommand{\letin}[3]{\sco{let } #1 = #2 \sco{ in } #3}
\newcommand{\fork}[1]{\sco{fork }#1}
\newcommand{\give}[2]{\sco{give } #1\ #2}
\newcommand{\take}[1]{\sco{take } #1}
\newcommand{\mkch}[1]{\sco{mkCh } #1}
\newcommand{\channel}[2]{\sco{Ch }#1\ #2}

\newcommand{\Unit}{\sco{Unit}}
\newcommand{\Chan}[1]{\sco{Chan }#1}

\title{Simply Typed Lambda Calculus with Channel}
\author{Tianyao Zhou (202101640)}

\begin{document}

\maketitle

\section{Project Description}

In this project, we formalize Simply Typed Lambda Calculus extended with asynchronized message-passing channels(referred as \lambdach below)
and formulate and prove progress and preservation theorem on both the term level and the configuration level.

Elements of \lambdach is inspired by Simon John Fowler\cite{fowler2019typed}.

\subsection{Syntax}

\begin{align*}
    \begin{array}[]{rrl}
        t & := & x                  \\
          &  | & \lam{x}{T}{t}      \\
          &  | & \unit              \\
          &  | & t\ t               \\
          &  | & \letin{x}{t}{t}    \\
          &  | & \fork{t}           \\
          &  | & \give{c}{t}        \\
          &  | & \take{c}           \\
          &  | & \mkch{T}           \\
          &  | & \channel{n}{T}
    \end{array}
\end{align*}

The first rule is variable $x$.

The second rule is abstraction on $x$ of type $T$ in term $t$.

The third rule is unit value.

The forth rule is let expression.

The fifth rule is fork expression, which fork a new thread.

The sixth rule is send expression, which sends $t$ to a channel $c$.

The seventh rule is receive expression, which receives an value from a channel $c$.

The eighth rule is channel creation expression, which create a new channel for communication.

The ninth rule is intermidiate representation of a channel on data of type $T$ and
a natural number identification $n$.

\subsection{Types}

\begin{gather*}
    \begin{array}[]{rrl}
        T & := & \Unit  \\
          &  | & T -> T \\
          &  | & \Chan{T}
    \end{array}
\end{gather*}

The first rule is \Unit\ type with only one inhabitant \unit.

The second rule is function type.

The third rule is channel type, which only carries data of type $T$.

\subsubsection{Typing Rules}

\begin{gather*}
    \inference{
        \Gamma[x] = T
    }{
        \Gamma |- x : T
    }\\
    \\
    \inference{
        x : T, \Gamma |- t : T'
    }{
        \Gamma |- \lam{x}{T}{t} : T -> T'
    }\\
    \\
    \inference{}{
        \Gamma |- unit : Unit
    }\\
    \\
    \inference{
        \Gamma |- t_1 : T & \Gamma |- t_2 : T -> T'
    }{
        \Gamma |- t_1\ t_2 : T'
    }\\
    \\
    \inference{
        \Gamma |- t_1 : T & x : T, \Gamma |- t_2 : T'
    }{
        \Gamma |- \letin{x}{t_1}{t_2} : T'
    }\\
    \\
    \inference{
        \Gamma |- t : Unit -> T
    }{
        \Gamma |- \fork{t} : Unit
    }\\
    \\
    \inference{
        \Gamma |- c : \Chan{T} & \Gamma |- t : T
    }{
        \Gamma |- \give{c}{t} : Unit
    }\\
    \\
    \inference{
        \Gamma |- c : \Chan{T}
    }{
        \Gamma |- \take{c} : T
    }\\
    \\
    \inference{}{
        \Gamma |- \mkch{T} : \Chan{T}
    }\\
    \\
    \inference{}{
        \Gamma |- \channel{n}{T} : \Chan{T}
    }
\end{gather*}

\subsection{Configuration}

The configuration consists of a list of channels and a list of threads. Each channel is a list of terms
and each thread is a term that's either under evaluation or a value.

    $$config := [c_0,\dots,c_n];[e_1,\dots,e_m]$$

\subsection{Operational Semantics}

\subsubsection{Evaluation Context}

The evaluation context is defined below.

\begin{gather*}
    \begin{array}{rrl}
        E & := & \circ  \\
          & |  & E\ t    \\
          & |  & t\ E    \\
          & |  & \letin{x}{E}{t} \\
          & |  & \fork{E} \\
          & |  & \give{E}{t} \\
          & |  & \give{t}{E} \\
          & |  & \take{E}
    \end{array}
\end{gather*}

And context filling rules.

\begin{gather*}
    \inference{}{
        e = \circ\{e\}
    }\\
    \\
    \inference{
        e_1 = E\{e\}
    }{
        e_1\ e_2 = (E\ t)\{e\}
    }\\
    \\
    \inference{
        value\ v_1 & e_2 = E\{e\}
    }{
        v_1\ e_2 = (v_1\ E)\{e\}
    }\\
    \\
    \inference{
        e_1 = E\{e\}
    }{
        \letin{x}{e_1}{t} = (\letin{x}{E}{t})\{e\}
    }\\
    \\
    \inference{
        e_1 = E\{e\}
    }{
        \fork{e_1} = (\fork{E})\{e\}
    }\\
    \inference{
        e_1 = E\{e\}
    }{
        \give{e_1}{e_2} = (\give{E}{e_2})\{e\}
    }\\
    \\
    \inference{
        value\ v_1 & e_2 = E\{e\}
    }{
        \give{v_1}{e_2} = (\give{v_1}{E})\{e\}
    }\\
    \\
    \inference{
        e_1 = E\{e\}
    }{
        \take{e_1} = (\take{E})\{e\}
    }
\end{gather*}

\subsubsection{Term-level rules}

\begin{gather*}
    \inference{
        value\ v
    }{
        \lam{x}{T}{t}\ v -> t[x/v]
    }\\
    \\
    \inference{
        value\ v
    }{
        \letin{x}{v}{t} -> t[x/v]
    }\\
    \\
    \inference{
        e -> e'
    }{
        E\{e\} -> E\{e'\}
    }
\end{gather*}

\subsubsection{Configuration-level rules}

\begin{gather*}
    \inference{}{
        [c_0,\dots,c_n];[fork (\lam{x}{\Unit}{t}),\dots,e_m] -> [c_0,\dots,c_n];[\unit,\dots,e_m,t]
    }\\
    \\
    \inference{
        value\ v
    }{
        [c_0,\dots,c_n];[\give{(\channel{p}{T})}{v},\dots,e_m] ->
        [c_0,\dots,c_p + [v],\dots,c_n];[\unit,\dots,e_m]
    }\\
    \\
    \inference{}{
        [c_0,\dots,c_p = [v]+c_p',\dots,c_n];[\take{(\channel{p}{T})},\dots,e_m] ->
        [c_0,\dots,c_p',\dots,c_n];[v,\dots,e_m]
    }\\
    \\
    \inference{
        e -> e'
    }{
        [c_0,\dots,c_n];[E\{e\},\dots,e_m] ->
        [c_0,\dots,c_n];[E\{e'\},\dots,e_m]
    }\\
    \\
    \inference{
        \lnot value\ e &
        [c_0,\dots,c_n];[e,\dots,e_m] ->
        [c_0',\dots,c_p'];[e',\dots,e_q]
    }{
        [c_0,\dots,c_n];[E\{e\},\dots,e_m] ->
        [c_0',\dots,c_p'];[E\{e'\},\dots,e_q]
    }\\
    \\
    \inference{
        value\ v
    }{
        [c_0,\dots,c_n];[v,e_1,\dots,e_m] ->
        [c_0,\dots,c_n];[e_1,\dots,e_m]
    }
\end{gather*}

\subsection{Progress and Preservation Theorem}

Progress at term level is formulated as below.

\begin{theorem}
    For all closed term $t$, if $t$ is of type $T$, then one of
    \begin{enumerate}
        \item $t$ is a value
        \item there exists a term $t'$, such that $t -> t'$
        \item there exists a evaluation context $E$, such one of
                \begin{enumerate}
                    \item there exists a type T, such that $t = E\{\mkch{T}\}$
                    \item there exists a value $c$, such that $t = E\{\take{c}\}$
                    \item there exists two values $c$ and $v$, such that $t = E\{\give{c}{v}\}$
                    \item there exists a value $f$, such that $t = E\{\fork{f}\}$
                \end{enumerate}
              holds
    \end{enumerate}
    holds.
\end{theorem}

Preservation at term level is formulated as below.

\begin{theorem}
    For all closed term $t$ of type $T$ and $t'$, if $t -> t'$, then $t'$ is also of type $T$.
\end{theorem}

Weak progress at configuration level is formulated as below.

\begin{theorem}
    For all configuration consists of a list of channels $chs$ and a list of threads $thrds$,
    if $thrds = t :: tail'$ and $t$ is a closed term of type $T$, and all channels refered in $t$ is
    in $chs$, then one of
    \begin{enumerate}
        \item $t$ is a value
        \item there exists a term $t'$, a list of channels $chs'$ and a list of threads $thrds'$
                such that $chs;thrds -> chs';t' :: thrds'$
        \item there exists a program context $E$, a channel $\channel{p}{T}$, such that
                if $t = E\{\take{(\channel{p}{T})}\}$, then $c_p = []$
    \end{enumerate}
\end{theorem}

Preservation theorem at configuration level is formulated as below.

\begin{theorem}
    For all configuration consists of a list of channels $chs$, two list of threads $thrds$ and $thrds'$ and a closed term $t'$,
    if $thrds = t :: tail'$ and $t$ is a closed term of type $T$ and $t$ is not a value, and all channels
    in $chs$ are the typed according to references in $t$, and $chs;thrds -> chs';t' :: thrds'$
    then $t'$ is also of type $T$.
\end{theorem}

\section{Problems}

\subsection{How to model the operational semantics of \lambdach?}

There are different ways of writing small-step semantics. One is structural operational semantics,
which captures the evaluation order of a certain type of expression by a series of rules that evaluates
sub-expressions.

% For example, a plus expression $plus\ e_1\ e_2$ is modeled by

% \begin{gather*}
%     \inference{
%         e_1 -> e_1'
%     }{
%         plus\ e_1\ e_2 -> plus\ e_1'\ e_2
%     }\\
%     \\
%     \inference{
%         value\ v_1 & e_2 -> e_2'
%     }{
%         plus\ v_1\ e_2 -> plus\ v_1\ e_2'
%     }\\
%     \\
%     \inference{
%         value\ v_1 & value\ v_2
%     }{
%         plus\ v_1\ v_2 -> v_1 + v_2
%     }
% \end{gather*}

Another one is contextual operational semantics, which captures evaluation order by evaluation contexts.

% We define evaluation context for plus expression as below.

% \begin{gather*}
%     \begin{array}[]{rrl}
%         E & := & \circ \\
%           &  | & plus\ E\ e \\
%           &  | & plus\ v\ E \\
%       \end{array}
% \end{gather*}

% Besides, we define a set of rules to fill the evaluation context.

% \begin{gather*}
%     \inference{}{
%         e = \circ\{e\}
%     }\\
%     \\
%     \inference{
%         e_1 = E_1\{e\}
%     }{
%         plus\ e_1\ e_2 = plus\ E_1\ e_2\{e\}
%     }\\
%     \\
%     \inference{
%         value\ v_1 & e_2 = E_2\{e\}
%     }{
%         plus\ v_1\ e_2 = plus\ v_1\ E_2\{e\}
%     }
% \end{gather*}

When writing operational semantics, we can have a general case,

\begin{gather*}
    \inference{
        e -> e'
    }{
        E\{e\} -> E\{e'\}
    }
\end{gather*}

% And a base case for plus,
and base cases for expressions in the language.

% \begin{gather*}
%     \inference{
%         value\ v_1 & value\ v_2
%     }{
%         plus\ v_1\ v_2 -> v_1 + v_2
%     }
% \end{gather*}

They are equivalent. But the latter has some superficial advantages. It makes writing rules simpler
and proof cleaner\cite{harper2016practical}.

I chose to model \lambdach with contextual operational semantics, because it allows one organize proof
in a modular manner and this is the method used in Fowler's paper\cite{fowler2019typed}.

\subsection{How to define configuration?}

In the original paper\cite{fowler2019typed}, the author used process calculi to represent evaluation configuration.
However, it's merely a way to represent threads and bind channel identification to a buffer(a list). Besides in the
paper\cite{fowler2019typed}, it also mentions canonical form of configuration, which is very similar to two list, one of which
represents channels and the other represents threads.

\subsection{How to capture concurrency?}

There are many ways to capture concurrency of programs. One is true concurrency, which allows multiple
threads to progress in one step. Another one is interleaved execution, which undeterministically select
one thread to progress in one step.

However, real-world concurrent programs always comes with a scheduler, which only allows some threads to
be executed at the same time and may or may not be biased. To make this project simpler, we only reduce the
first expression in the thread list and append the forked thread to the end. This is not concurrency in any
sense. We could improve on this, but I don't know how to do this. What I have in mind is that similar to
evaluation context of terms, we can have context for a list like $heads + [t] + tails$ saying that there are
some elements before $t$ represented by $heads$ and some elements after $t$ represented by $tails$.

\subsection{How to define contextual operational semantics in Coq?}

The main issue here is how and where we describe a term to be a value in evaluation contexts.

We can describe it as an inductive type and embed the type in the term definition.
When defining the structure of evaluation contexts, it naturally fits in. However, if one type of
expression could both be a value and a reducible term, for example the tuple expression, it will appear
both in the definition of values and terms. It seems a bit redundant.

We can also use a predicate to describe it. It is more general than the former method, and allows for
a cleaner definition of terms.

In this project, I chose the latter way.

If we don't embed value in evaluation contexts' definitioin, we need another way to say the evaluation context
is well-formed.

We can use a predicate for it. But we can also state it in context filling rules by saying that only
when the context is well-formed, we can fill it.

When proving theorems, we usually do an induction on evaluation context. I feel the former method may allow simpler
automation, while the latter requires \sco{inversion} tactic to recover information.

In this project, I chose the latter way because \lambdach is a rather small language.

\subsection{How to formulate progress and preservation theorem?}

Because we have message-passing communication in \lambdach, receiving from an empty channel could cause
evaluation to stuck and our type system is not powerful enough to guarantee progress property.
The progress rule statement in \textit{Programming Language Foundations} is not
suitable here.

At the beginning, we planned to model the language with structural operational semantics.
And Vlad sugguests that we could add a busy-waiting rule when evaluation gets stuck and state
progress the same way as STLC.

\begin{gather*}
    \inference{
        c_p = []
    }{
        [c_0,\dots,c_p,\dots,c_n];[\take{(\channel{p}{T})},e_1,\dots,e_m] ->
        [c_0,\dots,c_p,\dots,c_n];[\take{(\channel{p}{T})},e_1,\dots,e_m]
    }
\end{gather*}

However, this rule seems a bit weird. Now that I've chosen the contextual operational, which allows us to
pinpoint the next sub-epxression to be evaluated, we can state this as a special case in the progress theorem.
Clearly this progress now is a weak 

Alternatively, we can seperate progress theorem at term level and at configuration level by making all
concurrency-related expression special cases of progress theorem at term level. This also alows for cleaner
organization of proofs because we can use progress theorem at term level when proving progress theorem at
configuration level.

This is inspired by the paper\cite{fowler2019typed} this project based on.

\subsection{Information loss due to \sco{induction} tactic}

When doing induction on a term with complex sub term, \sco{induction} tactic may cause some information loss.
Usually, we can avoid information loss by substitute the complex term with a variable using \sco{remember} tactic, 
making the term we do induction on the most general form.

Besides there is a smarter tactic \sco{dependent induction}. However, I don't understand it so I don't use it.

In \textit{Logical Foundations}, it says
\begin{quote}
    One potentially confusing feature of the induction tactic is that it will let you try to
    perform an induction over a term that isn't sufficiently general.\cite{Pierce:SF1}
\end{quote}

In my case, it turns out that \sco{induction} need the term to be the most general form.

\section{Experience with Coq}

I really like the expressiveness of Coq. I can model inductively defined concepts effortlessly. It
also helps you keep track of hypotheses and goals when doing proofs, and it forces you to consider
all the cases, so you don't make mistakes. When modeling complicated problems, it is easy for people to
miss something in their proofs. However with Coq, either you model the problem in a wrong way or
you get stuck when proving. There is no tricks here.

However, in Coq, it is really tempting to blindly try \sco{inversion}, \sco{induction} or other tactics.
This is definitely not a good way of writing proofs, because if you modeled the problem erroneously, you
can easily waste tens of minutes on it. The better way is to examine the hypotheses and goals carefully
and see what you need to prove the goal and what is missing from the hypotheses. You need to understand
the problem before proving it. In many situations, you will need to define some lemmas to help you.
If you understand the problem, it is very natural to come up with one when you get stuck.

What I don't like Coq is that there are many house-keeping things you need to do when proving like
duplicating hypotheses, generalizing terms and remember terms before induction. And the standard library
is not as comfortable to use as Haskell's. It is not built around type classes.

\printbibliography

\end{document}